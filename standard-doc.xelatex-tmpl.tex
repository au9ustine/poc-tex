\documentclass[12pt,a4paper]{report}

%% 封面
\title{1003-测试流程检验}
\date{\today}
\author{au9ustine}

%% 引入包文件
\usepackage[utf8]{inputenc}     % 字符编码
\usepackage{fontspec}           % XeLatex 字体选择
\usepackage{xunicode}           % Unicode 支持
\usepackage{xltxtra}            % XeLaTex 额外自定义

%% 符号
\usepackage{amssymb}
\usepackage{amsmath}
\usepackage{graphicx}

%% 代码
\usepackage{listings}

%% 配置中文
\usepackage{xeCJK}                   % 中文支持
\setCJKmainfont{STSongti-SC-Regular} % macOS 宋体
%% \setCJKmainfont{PingFangSC-Regular}  % macOS 黑体
\xeCJKDeclareSubCJKBlock{Ext-A} { "3400 -> "4DBF }
\xeCJKDeclareSubCJKBlock{Ext-B} { "20000 -> "2A6DF }
\xeCJKDeclareSubCJKBlock{Kana}  { "3040 -> "309F, "30A0 -> "30FF, "31F0 -> "31FF, }
\usepackage[iso,english]{isodate}       % 使用ISO8601格式的日期
\usepackage{titlesec}           % 修改默认chapter/section等样式使之符合中文惯例
%\titlespacing*{\chapter}{0pt}{-50pt}{20pt}
\titleformat{\chapter}{\Huge\bfseries}{\thechapter}{1em}{}
\setlength{\parindent}{2em}     % 段落首行空 2 个字符
\usepackage{indentfirst}        % 文章首段首行缩进
\linespread{1.5}                % 行间距 1.5 倍

%% 书签
\usepackage[xetex,colorlinks,unicode]{hyperref}

%% 版面和边距
\usepackage{geometry}


%\renewcommand{\chaptername}{章节}

% 文档开始
\begin{document}

% 封面
\maketitle

% 目录
\renewcommand{\contentsname}{目录}
\tableofcontents

\chapter{适用范围}
该流程仅适用于\texttt{macOS}平台

\chapter{规范性引用}

\chapter{公式演示}

\[ \sum_{i=0}^n = \int f(x) dx \]

\chapter{中文段落}
孙子曰:兵者,国之大事,死生之地,存亡之道,不可不察也。

故经之以五事,校之以计,而索其情:一曰道,二曰天,三曰地,四曰将、五曰法。道者,令民与上同意也,故可以与之死,可以与之生,而不畏危。天者,阴阳,寒暑、时制也。地者,远近、险易、广狭、死生也。将者,智、信、仁、勇、严也。法者,曲制、官道、主用也。凡此五者,将莫不闻,知之者胜,不知者不胜。故校之以计,而索其情,曰:主孰有道?将孰有能?天地孰得?法令孰行?兵众孰强?士卒孰练?赏罚孰明?吾以此知胜负矣。

将听吾计,用之必胜,留之;将不听吾计,用之必败,去之。计利以听,乃为之势,以佐其外。势者,因利而制权也。

兵者,诡道也。故能而示之不能,用而示之不用,近而示之远,远而示之近;利而诱之,乱而取之,实而备之,强而避之,怒而挠之,卑而骄之,佚而劳之,亲而离之。攻其无备,出其不意。此兵家之胜,不可先传也。

夫未战而庙算胜者,得算多也;未战而庙算不胜者,得算少也。多算胜,少算不胜,而况于无算乎?吾以此观之,胜负见矣。

\chapter{代码演示}
\begin{lstlisting}
  N <- 365
\end{lstlisting}

\appendix
\renewcommand{\contentsname}{附录}
\setlength{\parindent}{0pt}
\chapter{修订记录}

% 文档结束
\rule[-0.3em]{0.38em}{1.1em}% Q.E.D
\end{document}
